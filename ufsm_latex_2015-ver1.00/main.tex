%%%% Arquivo base para o documento - ver. 1.00 (24/02/2016)
% % % % % % % % % % % % % % % % 
% % % % % % % % % % % % % % % % 
%%%%%%MDT UFSM 2015%%%%%%%%%%%%
% % % % % % % % % % % % % % % % 
% % % % % % % % % % % % % % % % 
% % %  OPCOES DE COMPILACAO  %%%%%%%%%%%


% % % % % PAGINACAO
% % % PAGINACAO SIMPLES (FRENTE): PARA TRABALHOS COM MENOS DE 100 PAGINAS
\documentclass[oneside,openright,12pt]{ufsm_2015} %%%%% OPCAO PADRAO -> PAGINACAO SIMPLES. PARA TRABALHOS COM MAIS DE 100 PAGINAS COMENTE ESTA LINHA E DESCOMENTE A LINHA 
% % % % % % % % % % % % % % % % % % % % % % % % % % % % % % % % % % % % % % %
% PAGINACAO DUPLA (FRENTE E VERSO): PARA TRABALHOS COM MAIS DE 100 PAGINAS
% \documentclass[twoside,openright,12pt]{ufsm_2015}  %%%% PARA TRABALHOS COM MAIS DE 100 PAGINAS DESCOMENTE AQUI
% % % % % % % % % % % % % % % % % % % % % % % % % % % % % % % % % % % % %

% % % % % % % % % % % % % % % % % % % % % % % % % % % % % % % % % % % % %
% % % % % % % % % % % % % % % % % % % % % % % % % % % % % % % % % % % % %
%%%%%%%%% DEFINICAO PADRAO DE PACOTES -- ALTERE POR SUA CONTA E RISCO
% % % % % % % % % % % % % % % % % % % % % % % % % % % % % % % % % % % % %

\usepackage{amsmath}
\usepackage{enumerate}
\usepackage{amssymb}
\usepackage{graphicx}
\usepackage{epsf,amsfonts}
\usepackage{amsfonts}
\usepackage{epstopdf}
\usepackage{float}
\usepackage{mhchem}
% % % %  PACOTE DE CODIFICACAO - PADRAO = UTF8
\usepackage[utf8]{inputenc}  %utf8
% \usepackage[latin1]{inputenc}   % europeu
% % % % % % % % % % % 
\usepackage[brazil]{babel}
\usepackage[T1]{fontenc}
\usepackage{indentfirst}
\usepackage{textcomp}
\usepackage{setspace}
\usepackage{picinpar}
\usepackage{ifthen}
\usepackage{path}
\usepackage{scalefnt}
\usepackage{tocloft}
\usepackage[overload]{textcase}

% % % % % % % % % % % % % % % % % % % % % % % % % % % % % % % % % % % % % % % % % % % 
% % % % % % % % FIM DA DEFINICAO PADRAO DE PACOTES  % % % % % % % % 
% % % % % % % % % % % % % % % % % % % % % % % % % % % % % % % % % % % % % % % % % % % 






% % % % % % % % PACOTES PESSOAIS % % % % % % % %  






% % % % % %  DEFINICOES PESSOAIS  








% % % % % % % % % % % % % % % % % % % % % % % % % % % % % % % % % % % % % % % % % % % 




% % % % % % % % % % % % % % % % % % % % % % % % % % % % % % % % % 
% % % % % % % % % % % % DADOS DO TRABALHO % % % % % % % % % % % % 
% % % % % % % % % % % % % % % % % % % % % % % % % % % % % % % % % 

% % % % % % % % % % INFORMACOES INSTITUCIONAIS % % % % % % % % % % 
% % CENTRO DE ENSINO DA UFSM
\centroensino{Centro de Tecnologia}  %%% NOME POR EXTENSO
\centroensinosigla{CT}  %%% SIGLA

% % CURSO DA UFSM
\nivelensino{Graduação}  %%%%%%% NIVEL DE ENSINO 
\curso{Sistemas de Informação}   %%%%% NOME POR EXTENSO
\ppg{PPGALGO}   %%%%%% SIGLA
\statuscurso{Curso}  %%%% STATUS= {Programa} ou {Curso}


% % % % % % % % % % INFORMACOES DO AUTOR % % % % % % % % % % 
\author{Marinara Rübenich Fumagalli}   %%%%% AUTOR DO TRABALHO
\sexo{F} %%%% SEXO DO AUTOR -> M=masculino   F=feminino (IMPORTANTE PARA AJUSTAR PAGINAS PRE-TEXTUAIS)
\grauensino{Graduação}    %%%%%%%% GRAU DE ENSINO A SER CONCLUIDO
\grauobtido{Bacharela}    %%%%% TITULO OBTIDO
\email{mrfumagalli@inf.ufsm.br}   %%%% E-MAIL PARA CATALOGRAFICA (COPYRIGHT) - OBRIGATORIO
%\endereco{Rua Mato Grosso, n. 119} %%%% TELEFONE PARA CATALOGRAFICA (COPYRIGHT) (CAMPO OPICIONAL -- CASO NAO POSSUA OU NAO QUEIRA DIVULGAR COMENTE A LINHA)
%\fone{55 99648 6140}   %%%% TELEFONE PARA CATALOGRAFICA (COPYRIGHT) FORMATO {11 2222 3333} (CAMPO OPICIONAL -- CASO NAO POSSUA OU NAO QUEIRA DIVULGAR COMENTE A LINHA)
%\fax{11 2222 3333}   %%%% FAX PARA CATALOGRAFICA (COPYRIGHT) FORMATO {11 2222 3333} (CAMPO OPICIONAL -- CASO NAO POSSUA OU NAO QUEIRA DIVULGAR COMENTE A LINHA)


% % % % % % % % % % INFORMACOES DA BANCA % % % % % % % % % % 
% OBSERVACOES: O CAMPO ORIENTADOR EH OBRIGATORIO E NAO DEVE SER COMENTADO
% % % % % %    OS DEMAIS MEMBROS DA BANCA (COOREIENTADOR E DEMAIS PROFESSORES) QUANDO COMENTADOS NAO APARECEM NA FOLHA DE APROVACAO (O LAYOUT DA FOLHA DE APROVACAO ESTA PREPARADO PARA O ORIENTADOR E ATE MAIS 4 MEMBROS NA BANCA
\orientador{Joaquim Vinicius Carvallho Assunção}{Dr}{UFSM}{M}{P}  %%%INFORMACOES SOBRE ORIENTADOR: OS CAMPOS SAO:{NOME}{SIGLA DA TITULACAO}{SIGLA DA INSTITUICAO DE ORIGEM}{SEXO} M=masculino   F=feminino {PARTE DA BANCA?} P=presidente  M=Membro  N=Nao faz parte
%\coorientador{Maria da Costa}{Dra}{AAAA}{F}{M} %%%INFORMACOES SOBRE CO-ORIENTADOR: OS CAMPOS SAO:{NOME}{SIGLA DA TITULACAO}{SIGLA DA INSTITUICAO DE ORIGEM}{SEXO} M=masculino   F=feminino {PARTE DA BANCA?} P=presidente  M=Membro  N=Nao faz parte
\bancaum{Banca Um}{Dr}{AAAA}{F}{M}  %%%INFORMACOES SOBRE PRIMEIRO NOME DA BANCA: OS CAMPOS SAO:{NOME}{SIGLA DA TITULACAO}{SIGLA DA INSTITUICAO DE ORIGEM}{SEXO} M=masculino   F=feminino {PARTE DA BANCA?} P=presidente  M=Membro  N=Nao faz parte

\bancadois{Banca Dois}{Dr}{BBBB}  %%%INFORMACOES SOBRE SEGUNDO NOME DA BANCA: OS CAMPOS SAO:{NOME}{SIGLA DA TITULACAO}{SIGLA DA INSTITUICAO DE ORIGEM}
\bancatres{Banca Três}{Dra}{CCCC} %%%INFORMACOES SOBRE TERCEIRO NOME DA BANCA: OS CAMPOS SAO:{NOME}{SIGLA DA TITULACAO}{SIGLA DA INSTITUICAO DE ORIGEM}
% \bancaquatro{Banca Quatro}{Dr}{DDDD} %%%INFORMACOES SOBRE QUARTO NOME DA BANCA: OS CAMPOS SAO:{NOME}{SIGLA DA TITULACAO}{SIGLA DA INSTITUICAO DE ORIGEM}
% \bancacinco{Banca Cinco}{Dra}{EEEE} %%%INFORMACOES SOBRE QUARTO NOME DA BANCA: OS CAMPOS SAO:{NOME}{SIGLA DA TITULACAO}{SIGLA DA INSTITUICAO DE ORIGEM}



% % % % % % % % % % INFORMACOES SOBRE O TRABALHO % % % % % % % % % %
% % % %  TITULO DO TRABALHO
\titulo{Aplicação de Redes Neurais para estimativa de temperaturas com base em amostras de foraminíferos} %% NAO EH NECESSARIO CAPITALIZAR
% % % %  TITULO DO TRABALHO EM INGLES
\englishtitle{Application of Neural Networks to estimate temperatures based on foraminifera samples}  %% NAO EH NECESSARIO CAPITALIZAR
% % % AREA DE CONCENTRACAO DO TRABALHO (CNPQ)
%\areaconcentracao{Inteligência Artificial}
% % % TIPO DE TRABALHO - MANTER APENAS UMA LINHA DESCOMENTADA
%\tese  %% Tese de <nivel de ensino>
% \qualificacao %% Exame de Qualificação de <nivel de ensino>
% \dissertacao %% Dissertacao de <nivel de ensino>
% \monografia %% Monografia
% \monografiag  %% Monografia (nao exibe area de concentracao)
% \tf  %% Trabalho Final de <nivel de ensino>
\tfg  %% Trabalho Final de Graduacao (nao exibe area de concentracao)
% \tcc  %% Trabalho de Conclusao de Curso
% \tccg  %% Trabalho de Conclusao de Curso (nao exibe area de concentracao)
% \relatorio  %% Relatório de Estágio (nao exibe area de concentracao)
% \generico   %%% Alternativa para aqueles cursos que nao recebem o titulo de bacharel ou licenciado. Ex: engenharia, arquitetura, etc... Os campos abaixo tambem devem ser preenchidos
%     \tipogenerico{Tipo de trabalho em português}
%     \tipogenericoen{Tipo de trabalho em inglês}
%     \concordagenerico{o}
%     \graugenerico{Engenheiro Eletricista}
% % % DATA DA DEFESA 
\data{08}{07}{2019} %% FORMATO {DD}{MM}{AAAA}



% % % % %  ALGUMAS ENTRADAS PRE-TEXTUAIS
% % % % CASO NAO QUEIRA UTILIZA-LAS COMENTE A LINHA DE COMANDO
% % % EPIGRAFE
\epigrafe{"Os animais são meus amigos e eu não como meus amigos"}{George Bernard Shaw}
\epigrafe{"Quando o homem aprender a respeitar até o menor ser da Criação, seja animal ou vegetal, ninguém precisará ensiná-lo a amar seu semelhante"}{Albert Schweitzer}
\epigrafe{“A natureza pode suprir todas as necessidades do homem, menos a sua ganância.”}{Mahatma Gandhi} %ESTRUTURA DE CAMPOS -> {Texto}{Autor}
% % % DEDICATORIA
\dedicatoria{Ao Rei da Espanha!}
% % % %  AGRADECIMENTOS
\agradecimentos{
A mim!
}

% % % % %  RESUMO E PALAVRAS CHAVE DO RESUMO - OBRIGATORIO PARA MDT-UFSM
\resumo{
Resumo aqui.
}
\palavrachave{Foraminíferos. Machine Learning. Paleoclima. Redes Neurais.}
% "... deverão constar, no mínimo, três palavras-chave, iniciadas em
% letras maiúsculas, cada termo separado dos demais por ponto, e
% finalizadas também por ponto." MDT 2012

% % % % %  ABSTRACT E PALAVRAS CHAVE DO RESUMO - OBRIGATORIO PARA MDT-UFSM
\abstract{
Abstract here.
}
\keywords{Foraminifera. Machine Leraning. Paleoclimate. Neural Networks.}


% % %  ATIVACAO DE LISTAS E PAGINAS ESPECIAIS
% % %  PARA QUE APARECAO NAO NO TEXTO DESCOMENTE A LINHA ABAIXO -> POR PADRAO TODAS ESTAO ATIVIDADAS

% % LISTA DE FIGURAS 
% \semfiguras   %%(QUANDO ATIVIDA NAO EXIBE A LISTA)
% % LISTA DE GRAFICOS 
% \semgraficos   %%(QUANDO ATIVIDA NAO EXIBE A LISTA)
% % LISTA DE ILUSTRACOES 
% \semilustracoes  %%(QUANDO ATIVIDA NAO EXIBE A LISTA)
% % LISTA DE TABELAS 
% \semtabelas   %%(QUANDO ATIVIDA NAO EXIBE A LISTA)
% % LISTA DE QUADROS 
% \semquadros   %%(QUANDO ATIVIDA NAO EXIBE A LISTA)
% % LISTA DE APENDICES 
% \semapendices  %%(QUANDO ATIVIDA NAO EXIBE A LISTA)
% % LISTA DE ANEXOS 
% \semanexos   %%(QUANDO ATIVIDA NAO EXIBE A LISTA)



% % % %  LISTA DE ABREVIATURAS E SIGLAS
%%%%%%%% OBS: O espaco entre colchetes \item[] e um ambiente matematico
%%%%%%%% para não utilizar comente as linhas abaixo.
\siglamax{SIGLAMAX} %%%% coloque aqui a maior sigla (indentacao)
\listadeabreviaturasesiglas{
\item[SIGLA1] Nome Completo da Sigla 1
\item[SIGLA2] Nome Completo da Sigla 2
\item[SIGLAMAX] Nome Completo da Sigla MAX
}

% % % %  LISTA DE SIMBOLOS
%%%%%%%% OBS: O espaco entre colchetes \item[] e um ambiente matematico
%%%%%%%% para não utilizar comente as linhas abaixo.
\simbolomax{(Re)2} %%%% coloque aqui o maior simbolo (indentacao)
\listadesimbolos{
\item[u_*]  Escala de velocidade de fricção 
\item[w_*]  Escala de velocidade convectiva
\item[(Re)^2] Maior simbolo da lista
}


% % FICHA CATALOGRAFICA
\semcatalografica  %%%%  (QUANDO ATIVIDA NAO EXIBE A FICHA CATALOGRAFICA NECESSITA DO ARQUIVO DA FICHA: ficha_catalografica.pdf
% % % A FICHA CATALOGRAFICA FORNECIDA PELA UFSM EH UM PDF DO TAMANHO A4
% % % OS COMANDOS ABAIXO DEFINEM AS MARGENS PARA CORTAR A FICHA FORNECIDA E COLOCA-LA COMO UMA FIGURA NO DOCUMENTO LATEX
\margemesquerda{4}   %%%% CORTE DE MARGEM ESQUERDA EM CM
\margemdireita{1.5}   %%%% CORTE DE MARGEM DIREITA EM CM
\margemsuperior{17}  %%%% CORTE DE MARGEM SUPERIOR EM CM
\margeminferior{3} %%%% CORTE DE MARGEM INFERIOR EM CM
% % %  DICA: IMPRIMA UMA COPIA DA FICHA CATALOGRAFICA E FACA A MEDIDA DAS MARGENS!




% % FOLHA DE ERRADA (versao rudimentar...pode ser aprimorado)
% % para não utilizar comente as linhas abaixo.
% % deve ser preenchida como um ambiente tabular de quatro colunas:
% % pagina & linha & onde se le & leia-a se \\
%\errata{
%10   &    10    & errado   & certo \\
%\hline
%12    &    5     & errado com um texto mais longo & certo agora com um texto mais longo\\
%\hline
%13   &    3    & $x^2$   & $2x$\\
%}
% % % % % % % % % % % % % % % % % % % % % % % % % % % % % % % % % % % % % % % % % % % % % % 


% % % % % % % % % % % % % % % % % % % % % % % % % % % % % % % % % % % % % % 
% % % % % % % % % % % %  OPCOES DE FORMATACAO % % % % % % % % % % % % % % %
% % % % % % % % % % % % % % % % % % % % % % % % % % % % % % % % % % % % % % 
% % % CAPITULO: por padrao alinhado a esquerda. Para ativar alinhamento centralizado descomente o comando abaixo

%\centralizado  %%%% <<< centraliza todos os capitulos

% % % % % % % % % % % % % % % % % % % % % % % % % % % % % % % % % % % % % %
% % % FONTES: descomente uma das opcoes. caso nenhuma seja ativada a clase usara a fonte padrao do latex

%% helvetica
\usepackage[scaled]{helvet}
\renewcommand*\familydefault{\sfdefault}

%% arial
% \renewcommand{\rmdefault}{phv} % Arial
% \renewcommand{\sfdefault}{phv} % Arial

%%times
% \usepackage{mathptmx}

% % % % % % % % % % % % % % % % % % % % % % % % % % % % % % % % % % % % % % 
% % % % % % % % % % % % % % % % % % % % % % % % % % % % % % % % % % % % % % 
% % % % % % % % % % % % % % % % % % % % % % % % % % % % % % % % % % % % % % 
% % % % % % % % % % % % % % % % % % % % % % % % % % % % % % % % % % % % % % 


% % % % % % % % % % % % % % % % % % % % % % % % % % % % % % % % % % % % % % 
% % % % % % % % % % % % % % % % % % % % % % % % % % % % % % % % % % % % % % 
% % % % % % % % % % % %  INICIO DO DOCUMENTO  % % % % % % % % % % % % % % %
% % % % % % % % % % % % % % % % % % % % % % % % % % % % % % % % % % % % % % 
% % % % % % % % % % % % % % % % % % % % % % % % % % % % % % % % % % % % % %


\begin{document}



% % % % % % % % % % % % % % % % % % % % % % % % % % % % % % % % % % % % % % 
\pretextual  %%%% GERA AS PAGINAS PRE-TEXTUAIS   
% % % % % % % % % % % % % % % % % % % % % % % % % % % % % % % % % % % % % % 

% % % % % % % % % % % % % % % % % % % % % % % % % % % % % % % % % % % % % % 
% % % % % CORPO DO TRABALHO - INCLUA OS SEUS TEXTOS AQUI
% % % % % SUGESTAO -> UTILIZE ARQUIVOS EXTERNOS A PARTIR DO COMANDO \input




% % % % % % % % % % % % % % % % % % % % % % % % % % % % % % % % % % % % % % 
% % % % % % % % % % INICIO DAS PAGINAS TEXTUAIS % % % % % % % % % % % % % % 
% % % % % % % % % % % % % % % % % % % % % % % % % % % % % % % % % % % % % % 




% % % % % % % % % % % % % % % % % % % % % % % % % % % % % % % % % % % % % % 
% % % % % % % % % % % % % INTRODUCAO % % % % % % % % % % % % % % % % % % % % 
% % % % % % % % % % % % % % % % % % % % % % % % % % % % % % % % % % % % % % 
\introducao{%
\par Insira aqui a introdução!!!
\par Insira aqui a introdução!!!

\par Insira aqui a introdução!!!

\par Insira aqui a introdução!!!

\par Insira aqui a introdução!!!

\par Insira aqui a introdução!!!

\par Insira aqui a introdução!!!

\par Insira aqui a introdução!!!

\par Insira aqui a introdução!!!

\par Insira aqui a introdução!!!

\par Insira aqui a introdução!!!

\par Insira aqui a introdução!!!

\par Insira aqui a introdução!!!

\par Insira aqui a introdução!!!

\par Insira aqui a introdução!!!

\par Insira aqui a introdução!!!

\par Insira aqui a introdução!!!

\par Insira aqui a introdução!!!

\par Insira aqui a introdução!!!

\par Insira aqui a introdução!!!

\par Insira aqui a introdução!!!

\par Insira aqui a introdução!!!

\par Insira aqui a introdução!!!

\par Insira aqui a introdução!!!
\par Insira aqui a introdução!!!

\par Insira aqui a introdução!!!

\par Insira aqui a introdução!!!

\par Insira aqui a introdução!!!

\par Insira aqui a introdução!!!

\par Insira aqui a introdução!!!

\par Insira aqui a introdução!!!

\par Insira aqui a introdução!!!

\par Insira aqui a introdução!!!

\par Insira aqui a introdução!!!

\par Insira aqui a introdução!!!

\par Insira aqui a introdução!!!

\par Insira aqui a introdução!!!

\par Insira aqui a introdução!!!

\par Insira aqui a introdução!!!

\par Insira aqui a introdução!!!

\par Insira aqui a introdução!!!

\par Insira aqui a introdução!!!

\par Insira aqui a introdução!!!

\par Insira aqui a introdução!!!

\par Insira aqui a introdução!!!

\par Insira aqui a introdução!!!

\par Insira aqui a introdução!!!
\par Insira aqui a introdução!!!

\par Insira aqui a introdução!!!

\par Insira aqui a introdução!!!

\par Insira aqui a introdução!!!

\par Insira aqui a introdução!!!

\par Insira aqui a introdução!!!

\par Insira aqui a introdução!!!

\par Insira aqui a introdução!!!

\par Insira aqui a introdução!!!

\par Insira aqui a introdução!!!

\par Insira aqui a introdução!!!

\par Insira aqui a introdução!!!

\par Insira aqui a introdução!!!

\par Insira aqui a introdução!!!

\par Insira aqui a introdução!!!

\par Insira aqui a introdução!!!

\par Insira aqui a introdução!!!

\par Insira aqui a introdução!!!

\par Insira aqui a introdução!!!

\par Insira aqui a introdução!!!

\par Insira aqui a introdução!!!

% \par Insira aqui a introdução!!!

}%
% % % % % % % % % % % % % % % % % % % % % % % % % % % % % % % % % % % % % % 
\geraintro  %%%% GERA INTRODUCAO   % % % % % % % % % % % % % % % % % % % % % 
% % % % % % % % % % % % % % % % % % % % % % % % % % % % % % % % % % % % % % 
% % % % % % % % % % % % % % % % % % % % % % % % % % % % % % % % % % % % % % 

\chapter{Algumas notas sobre bibliografia}


\par O arquivo \textit{referencias.bib} é o nome padrão das referências para este modelo. Para alterá-lo modifique o argumento entre chaves na definição de bibliografia.

   \section{Modelos de citação}
      \par No arquivo modelo de referências, também existem alguns exemplos de diferentes classes de citações. Todas elas podem ser usadas com o \textit{$\backslash$cite$\{$label$\}$} ou \textit{$\backslash$citeonline$\{$label$\}$}, dependendo da forma de citação\footnote{Este é um teste de nota de rodapé}.

  \par Além disso, pode-se incluir obras na bibliografia que nortearam o trabalho mesmo que elas não apareçam diretamente no texto, utilizando o comando \textit{$\backslash$nocite$\{$label$\}$}. Além disso, pode-se citar vários trabalhos em conjunto, por exemplo:
  
  \begin{center}\rule{0.5\textwidth}{1pt}\\$\backslash cite\{label1,label2,label3,...\}$\end{center}

  \begin{verbatim}
  Os ventos do norte não movem moinhos \cite{tcc:mintegui2014, 
  diss:anabor2004, tese:anabor2008, livro:halliday28ed, 
  livro:fedorova:v1, site:amsglo:fog}.
  \end{verbatim}
      
  \par Os ventos do norte\footnote{Este é um teste de nota de rodapé} não movem moinhos \cite{tcc:mintegui2014,diss:anabor2004,tese:anabor2008,livro:halliday28ed,livro:fedorova:v1,site:amsglo:fog}.
        
        \begin{center}\rule{0.5\textwidth}{1pt}\\$\backslash citeonline\{label1,label2,label3,...\}$\end{center}
        
  \begin{verbatim}
  Segundo \citeonline{cap:livro:djuric1994, livro:aris1989, 
  livro:cotton1989, cap:livro:wyngaard1981, livro:fedorova:v1, 
  artigo:fujita1981,puhalesBLT2010,artigo:janjic2002,
  cbmet:anabor2012,site:wrfhome}, os ventos do norte não movem moinhos.
  \end{verbatim}
            
        \par Segundo \citeonline{cap:livro:djuric1994, livro:aris1989, livro:cotton1989, cap:livro:wyngaard1981, livro:fedorova:v1, artigo:fujita1981,puhalesBLT2010,artigo:janjic2002,cbmet:anabor2012,site:wrfhome}, os ventos\footnote{Este é um teste de nota de rodapé} do norte não movem moinhos.
        \begin{center}\rule{0.5\textwidth}{1pt}\end{center}   
         \par OBS: Trabalhos com mais de três autores aparecerão com a abreviatura ``et al'' no corpo do texto, porém todos os nomes da bibliografia (norma da UFSM que é definida nas opões do abntcite nas definições do arquivo). 
  
  \subsection{O comando \textit{apud}}
  \par A citação \textit{apud} ocorre quando você cita algum autor através de outra obra, sem ter consultado-a propriamente. Neste caso a citação é feita da seguinte forma:
  \begin{center}
  \rule{0.5\textwidth}{1pt}\\
  $\backslash apud\{material\_lido\}\{material\_citado\_no\_material\_lido\}$ \\
  \end{center}
\begin{verbatim}
Sobre a circulação geral da atmosfera pode-se dizer que os ventos do norte
não movem moinhos \apud{livro:monin:v1}{apud:richardson1922}.
\end{verbatim}
  
  Sobre a circulação geral da atmosfera pode-se dizer que os ventos do nortenão movem moinhos \apud{livro:monin:v1}{apud:richardson1922}.
\begin{center}\rule{0.5\textwidth}{1pt}\end{center}  
  \par Nesse caso, na bibliografia só constará a obra consultada e não aquele referenciada pela obra. Para que isso ocorra naturalmente, a obra consultada deve ser incluída normalmente no arquivo referencias.bib enquanto a obra referenciada indiretamente deve ser incluída com a opção \textit{@hidden}, conforme o modelo de referências.

      \subsubsection{\textit{Apud on line}}
      
      \par O \textit{apudonline} se aplica da mesma maneira que o \textit{apud} descrito anteriormente. O termo \textit{on line} é alusivo ao \textit{$\backslash$citeonline$\{$label$\}$} definido no abntex. Nesse caso a citação é feita da seguinte forma:
      \begin{center}
      \rule{0.5\textwidth}{1pt}
            $\backslash apudonline\{material\_lido\}\{material\_citado\_no\_material\_lido\}$ \\
      \end{center}

 \begin{verbatim}
Segundo \apudonline{livro:monin:v1}{apud:richardson1922}, os ventos do
norte não movem moinhos.
\end{verbatim}

            Segundo \apudonline{livro:monin:v1}{apud:richardson1922}, os ventos do norte não movem moinhos.

       \paragraph{Teste de seção quinária}
       
       
       \par Texto texto texto.
       
       
         \chapter{Quadros, tabelas e figuras}
         
         \par Na MDT da UFSM há uma clara diferença entre tabelas e quadros, quanto a sua apresentação. Aqui, para inserir tabelas usa-se o ambiente tradicionalmente definido \textit{table}. A partir deste modelo simples:
         

        
        
         \begin{verbatim}
    \begin{table}[ht]
    \centering
    \caption{Modelo de tabela para MDT-UFSM.}
    \begin{tabular}{ c c c }
    \hline
    Abacate & Banana & Canela \\
    \hline
    21 & 34 & 56 \\
    -3 & 245 & 23 \\
    -25 & -0,57 & 2 \\
                \hline
                 \end{tabular}
                \vspace{\baselineskip} %%% linha em branco para atendender a norma
    \fonte{Adaptado de \citeonline{livro:halliday28ed}.}
    \end{table}
    \end{verbatim}
         
         \noindent resulta:
         
         \begin{table}[ht]
         \centering
         \caption{Modelo de tabela para MDT-UFSM.}
         \begin{tabular}{ c c c }
         \hline
         Abacate & Banana & Canela \\
         \hline
         21 & 34 & 56 \\
         -3 & 245 & 23 \\
         -25 & -0,57 & 2 \\
         \hline
         \end{tabular}
         \vspace{\baselineskip} %%% linha em branco para atendender a norma
          \fonte{Adaptado de \citeonline{livro:halliday28ed}.}
         \end{table}
         
         \par Note que, adicionalmente, foi definido um comando novo: ``fonte''. Ele serve para indicar a fonte da tabela quando necessário, mas também pode ser usado em outros ambientes.
         
         \par Para inserir quados, foi criado um novo ambiente: ``quadro''. Ele seve ser usado de forma semelhante a tabela, como o ambiente tabular. Contudo, neste caso, as linhas verticais e horizontais estão sempre presentes. Um exemplo simples é o seguinte: 
         
         
         \begin{verbatim}
      \begin{quadro}
        \caption{Modelo de quadro para MTD-UFSM.}
      \centering
      \begin{tabular}{| c |c |c }
      \hline
      Abacate & Banana & Canela \\
      \hline
      21 & 34 & 56 \\
      \hline
      -3 & 245 & 23 \\
      \hline
      -25 & -0,57 & 2 \\
      \hline
      \end{tabular}
      \vspace{\baselineskip} %%% linha em branco para atendender a norma
      \fonte{Adaptado de \citeonline{livro:halliday28ed}.}
      \end{quadro}
         \end{verbatim}
         
         \noindent resultando:
         
      \begin{quadro}
      \caption{Modelo de quadro para MTD-UFSM.}
      \centering
      \begin{tabular}{| c |c |c |}
      \hline
      Abacate & Banana & Canela \\
      \hline
      21 & 34 & 56 \\
      \hline
      -3 & 245 & 23 \\
      \hline
      -25 & -0,57 & 2 \\
      \hline
      \end{tabular}
      \vspace{\baselineskip} %%% linha em branco para atendender a norma
          \fonte{Adaptado de \citeonline{livro:halliday28ed}.}
      \end{quadro}
      

         \noindent Assim como para as tabelas e figuras, já está definida uma lista de quadros. Além disso, o comando ``fonte'' também pode ser usado aqui se necessário, bem como para figuras. Vale lembrar que, na MDT-UFSM, as legendas de quadros e figuras aparecem embaixo das mesmas. Já nas tabelas, em cima. A fonte sempre embaixo.
         
         \par As figuras devem ser inseridas com o ambiente padrão: \textit{figure}. Veja um exemplo simples:
         
         \begin{verbatim}
      \begin{figure}[ht]
    \centering
    \includegraphics[width=0.6\textwidth]{figuras/pretextuais.png}
    \caption{\label{exepretex} Sequência dos elementros pré-testuais da MDT-UFSM}
    \vspace{\baselineskip} %%% linha em branco para atendender a norma
    \fonte{Adaptado de \citeonline{man:MDTUFSM2012}.}
      \end{figure}
         \end{verbatim}
         
         \begin{figure}[ht]
          \caption{\label{exepretex} Sequência dos elementros pré-testuais da MDT-UFSM}
      \centering
      \includegraphics[width=0.6\textwidth]{figuras/pretextuais.png}
      \vspace{\baselineskip} %%% linha em branco para atendender a norma
            \fonte{Adaptado de \citeonline{man:MDTUFSM2012}.}
         \end{figure}
        
        
         
\chapter{Conclusão}

  \par Conclusão do trabalho. 

         \begin{verbatim}
      \begin{ilustracao}[ht]
    \centering
    \includegraphics[width=0.6\textwidth]{figuras/pretextuais.png}
    \caption{\label{exepretex} Sequência dos elementros pré-testuais da MDT-UFSM}
    \vspace{\baselineskip} %%% linha em branco para atendender a norma
    \fonte{Adaptado de \citeonline{man:MDTUFSM2012}.}
      \end{ilustracao}
         \end{verbatim}
         
         \begin{ilustracao}[ht]
          \caption{\label{exepretex} Sequência dos elementros pré-testuais da MDT-UFSM}
      \centering
      \includegraphics[width=0.6\textwidth]{figuras/pretextuais.png}
      \vspace{\baselineskip} %%% linha em branco para atendender a norma
            \fonte{Adaptado de \citeonline{man:MDTUFSM2012}.}
         \end{ilustracao}
% % % % % % % % % % % % % % % % % % % % % % % % % % % % % % % % % % % % % % 
% % % % % % % % % % % % FIM DAS PAGINAS TEXTUAIS % % % % % % % % % % % % % % 
% % % % % % % % % % % % % % % % % % % % % % % % % % % % % % % % % % % % % % 

%CITAR P/ REFERENCIAR

  \citeonline{tcc:gpereira2011, diss:cerentini2018, diss:dias2014,
  diss:goulart1999, diss:friedrichs2013, tese:pereira2017,
  tese:braga2014, tese:padilha2014, tese:ferraz2013,
  livro:petro2019, livro:petro2018, livro:haykin1994,
  livro:bragacarvalholudemir1998, livro:cushman1948,
  livro:zuurmeestersieno2009, livro:kovacz2006, livro:gurney1997,
  livro:tafnerxerezfilho1995, livro:infante1988,
  livro:haykin2001, livro:silvaspattiflauzino2010,
  livro:sariloesch1996, livro:kovacz1997,
  livro:facelilorenagamacarvalho2011,
  cap:livro:hastietibshiranifriedman2009,
  artigo:saad1996,
  site:ufrgsfora}.


% % % % % % % % % % % % % % % % % % % % % % % % % % % % % % % % % % % % % %   
% % % % % % % % % % % % % BIBLIOGRAFIA  % % % % % % % % % % % % % % % % % % 
% % % % % % % % % % % % % % % % % % % % % % % % % % % % % % % % % % % % % %   

\bibliografia{referencias}  %%%%% BIBLIOGRAFIA -> INCLUIR NAS CHAVES O NOME DO ARQUIVO *.BIB  
  
  
  
% % % % % % % % % % % % % % % % % % % % % % % % % % % % % % % % % % % % %   
% % % % % % % % % % % % % APENDICES % % % % % % % % % % % % % % % % % % %
% % % % % % % % % % % % % % % % % % % % % % % % % % % % % % % % % % % % %   
  \apendice %%%% TEXTOS A PARIR DESTE PONTO SERAO CONSIDERADOS APENDICES

\chapter{Demonstração de algo}
        \par Algo como apêndice.  

        
        
        
% % % % % % % % % % % % % % % % % % % % % % % % % % % % % % % % % % % % % %   
% % % % % % % % % % % % % % % ANEXOS  % % % % % % % % % % % % % % % % % % % 
% % % % % % % % % % % % % % % % % % % % % % % % % % % % % % % % % % % % % %   
        \anexo    %%%% TEXTOS A PARIR DESTE PONTO SERAO CONSIDERADOS ANEXOS
        
\chapter{Algo interessante que alguém fez}
         \par Algo como anexo.
         
          \begin{verbatim}
      \begin{ilustracao}[ht]
    \centering
    \includegraphics[width=0.6\textwidth]{figuras/pretextuais.png}
    \caption{\label{exepretex} Sequência dos elementros pré-testuais da MDT-UFSM}
    \vspace{\baselineskip} %%% linha em branco para atendender a norma
    \fonte{Adaptado de \citeonline{man:MDTUFSM2012}.}
      \end{ilustracao}
         \end{verbatim}
         
         \begin{grafico}[ht]
          \caption{\label{exepretex1} Sequência dos elementros pré-testuais da MDT-UFSM}
      \centering
      \includegraphics[width=0.6\textwidth]{figuras/pretextuais.png}
      \vspace{\baselineskip} %%% linha em branco para atendender a norma
            \fonte{Adaptado de \citeonline{man:MDTUFSM2012}.}
         \end{grafico}
         
         
         \section*{Teste de seção dentro do anexo}


\end{document}